\documentclass[conference]{IEEEtran}
\IEEEoverridecommandlockouts

\usepackage{cite}
\usepackage{amsmath,amssymb,amsfonts}
\usepackage{algorithmic}
\usepackage{graphicx}
\usepackage{textcomp}
\usepackage{xcolor}
\def\BibTeX{{\rm B\kern-.05em{\sc i\kern-.025em b}\kern-.08em
    T\kern-.1667em\lower.7ex\hbox{E}\kern-.125emX}}

\begin{document}

\title{Tomato leaf Disease detection using TensorFlow and  Keras libraries}

\author{
\IEEEauthorblockN{1\textsuperscript{st} Rishabh kushwaha}
\IEEEauthorblockA{\textit{Department of Computer Science and Engineering} \\
\textit{Sharda university} \\
Greater Noida, India \\
rishabhkushwaha9559@gmail.com}
\and
\IEEEauthorblockN{2\textsuperscript{nd} Vishal Jain}
\IEEEauthorblockA{\textit{Department of Computer Science and Engineering} \\
\textit{Sharda University} \\
Greater Noida, India \\
vishal.jain@sharda.ac.in}
}

\maketitle

\begin{abstract}
In India around 58 percent of India's population depend on agriculture for their livelihood, still our Indian farmers do not earn enough to live their livelihood, they suffer more than other occupations. To improve their livelihood we must improve their income, to increase income we must increase crop production. To increase crop production we must prevent disease, but Indian farmers use traditional process to cultivate the crops, they must use smart technology to improve the crop production. Modern technology helps to monitor every stage of their crop growth, health, moisture, nutrition and control sun-light. In this paper, we introduce our tomato plant disease detection model to detect diseases like septoria spot, late blight, mosaic virus, bacterial spot and so many. To prevent spreading in the whole crop field we must identify at the initial stage of disease. To detect the disease use modern technology. Here in this paper to train our model we used 10,000 tomato leaf. After that we divide into 3 part, 80\% training, 20\% validation and 20\% for testing. After training the model achieved around 95.68\% accuracy rates, testing accuracy 88.9\% and validation accuracy 88.81\%. The total size of the train model is around 254MB .
\end{abstract}

\begin{IEEEkeywords}
Tomato leaf disease, Tensorflow, Keras, CNN model, Crop leaf detection, farmers
\end{IEEEkeywords}

\section{Introduction}
Tomatoes are a crucial part of human daily diets worldwide, valued for their culinary versatility and nutritional benefits including vitamins, antioxidants and minerals. Tomato originated in western South American with domestication traced back to ancient civilization in Mexico[1]. Tomato grow strongly and vigorously in warm , sunney environments with well drained and moisture without waterlogging, fertile soil, temperature around 21 to 29 celsius, soil pH value around 6.0 to 6.8, sunlight around 6 to 8 hours daily, tomatoes optimal storage temperature 4-10 celsius and storage life 4-7 days[2]. Tomatoes are easily influenced by various diseases caused by bacteria, fungi, environment and viruses. Here in this paper discussed some pathogenic disease septoria leaf spot, late blight and early blight. Septoria leaf Vishal Jain Department of Computer Science and Engineering, Sharda School of Engineering and Technology Sharda University ,Greater Noida ,India Email: vishal.jain@sharda.ac.in spot has symptoms of small, dark circular spots that have yellow holes around and they appear on lower parts of leaves. Late blight affect larger portion of leaves that look white, fuzzy on the underside leaf and destroy crops . Early blight appear on leaf and stems that are identified by large target like rings. These diseases can be spread by many ways like Moisture on leaves provide great conditions for spores of these diseases. To prevent it, use a drop irrigation system and watering at morning time for dry excess water from the plant root [3]. Some common diseases appear on tomato plants . Fig. 1 (a) Septoria leaf spot (b) Late blight (c) Early blight [4] Deep learning model is used to train neural network. The primary identification and category of tomato leaf disease is helping farmers to secure their crops from disease. These model analyze photo of tomato leaf plants to recognize disease symptom, enabling quick and effectively mediate. The aim of the project to create deep learning model using keras. Keras is a library or framework used to build and train neural network, it provide build in module for machine learning platform and neural learning computations. These networks have the ability to extract characteristic from the input images and have an autodidact process. To Improve the picture data, pre processing techniques such as edge preserving, smoothing, color, noise reduction, and contrast ameliorate are used. Keras is used to load and process pictures to build images, specify model by defining labels for various classes in the dataset, loading the data, converting and resizing the picture. then use the model to make forecast about the images (Tomato leaf disease ) . For the project experiment, an online dataset which is available on kaggle that have various types of tomato disease leaf, 80 percent data has been used for training, 20 percent used for testing and 20 percent used in validation. The proposed model has 95.68 percent accuracy result [5][6] . This model used to identify tomato leaf disease early to reduce for the spreading. Here we used 10,000 images of tomato leaf include 9,000 disease leaf and 1,000 healthy leaf

\section{Literature Review}
Heba Al-Hiary and colleagues introduced a plant disease detection method utilizing k-means clustering combined with a Neural Network. This model effectively identifies and classifies plant diseases, achieving an accuracy range of 83\% to 94\%. The method's strengths include precise disease detection with minimal computational effort, though its recognition rate may decline over time[7]. In paper [8] the author developed a CNN model and deployed it on an android and their model based on tomato leaves disease detection. When users scan the image of leaves then it provide information about disease. Advancing agriculture through cutting\_edge deep learning technique for plant disease, the propose method achieve result sensitivity 97.32\%, specificity 97.48\% and accuracy be 98.78\% [9] . H.Sabril and his companion K.Satish used supervised learning techniques for training models. Classification accuracy of tomato plants disease 97.3\% and 98.703 gini index based root node. While the model demonstrates good accuracy, newer classification techniques could potentially enhance its performance.[10] Paper [11] focuses on identify tomato leaves diseases using deep neural network through meta learning, utilizing pretrained models like AlexNet, GoogLeNet, and ResNet. The study found that ResNet optimized with SGD achieve around 96.51\% of accuracy, with further refinements resulting in a precision of 97.28\%. Dilated convolution, The structure of DCNet consists of a convolutional block with dual 3×3 convolutions (32 filters each), followed by max-pooling, dropout layers and a dilated convolutional block with two 3×3 dilation convolutions (64 filters, dilation factor d=2). DCNet's effectiveness was demonstrated through experiments on the PlantVillage dataset, which includes images of ten different tomato leaf diseases. The model achieved a superior test accuracy of 99.03\% compared to several established methods, including VGG16, VGG19, ResNet18, GoogLeNet and AlexNet, showcasing its robust performance[12]. In paper[13] authors propose their CNN model using 10 different groups of leaf and they achieve an accuracy of 93\%. Stanley and his associate develop model to detect disease through photo and video of its leaves using YoloV8 and RoboFlow with average precision 99\% and recall rate of 91.9\% [14]. Mahsa, Aliakbar and Mohsen from Iran,They use CCMT dataset and perform model training by VGG19 and achieved accuracy 92.76\%, precision 92.74\%, recall 95.09\% and F1-score 90.86\%. The performance of ResNet-101 and MobileNet-v2 significantly degraded [15]. In paper [16] Bibek, Sandhya and Charu, they use KNN classifier extraction feature for tomato leaf diseases model. Their model extracted a combination of textural, statistical and morphological features. They achieved a high accuracy of 93.5\%. Hritwik and his companion diagnosing the tomato leaf disease by using ResNetl52V2, ResNet50, VGG19 and Inception v2. They use 20,000 picture for training model, they achieve around training accuracy 96.91\%, validation 92.72\% and loss 3.6172 from ResNetl52V2 [17]. In paper[18] authors introduce new ways to use model and robotic hand. They combine their model with a robot after they scan the field and detect the plant. Their accuracy was 98.27\%, testing 91.01, validation accuracy 90.08\%. Matas and Tomyslav from lithuania, They identify tomato leaves disease using Yolo v8 detection model on gpu and raspberrayPi, input size be 640 x 640 px and 100 epochs. Their small model predicts 0.78 precision, 0.75 recall and large model 0.79, 0.81 precision and recall [19].

\section{Methodology}
Here some steps were perform to train our model for tomato leaf disease detection. Phase A: Data Collection and Preprocessing The first step is gathering a comprehensive dataset for model training of tomato leaf disease detection. The dataset is a collection of healthy tomato leaf photo and unhealthy leaf images. Over here is a collection of photo taken from kaggle publicly available dataset. For training a total 10,000 images of tomato leaves, including 9,000 unhealthy and 1,000 healthy tomato leaf photos. Fig 2. Datasets of Tomato leaf [4] When images are collected then our next step is to perform Phase C: Training the CNN Model several preprocessing steps to assemble the data for model training. This be composed of following: 1) Resizing : All pictures should be of uniform dimension of 224x224 pixel to normalize input size of image for cnn model. 2) Augmentation: To augment the robustness of the model and prohibit the overfitting so apply a data augmentation approach- shear range=0.2, zoom\_range=0.2 and horizontal\_flip=true. Convolutional Neural Network (CNN) model for recognizing and classifying different types of tomato leaf diseases. In this step, the model is taken labeled images one by one and internal parameters of it was fine-tuned using errors in predictions until a level of highest accuracy is reached after some iterations[22]. 80 percentage dataset used for model training, 20 percentage for testing the model and 20 percentage for validation of the model. Some following step are used for training[23]. 1) Batch Processing of Data: To make sense out of the data we generally divide all those dataset into set batches and that forms our training process. Rather than one shot processes all the images, these batches (usually 32 or 64 images each depending on how much computational power you have) are supplied to the model. This method called mini-batch gradient descent, improves on the memory efficiency and results in a more reliable training as well. Our model given batch size is 32. 2) Pass over the Network(Process) or Forward pass: The model forward passes its layers for every batch of images. In the beginning, images come into Conv layers which move a filter around pixels to find vital information like edge detection and texturing(color) making some kind of pattern. After that, the features are further condensed with pooling layers reducing data dimensionality leading to a more efficient model while keeping most important details. Once the data passes through convolutional and pooling layers, we flatten it into a single vector which goes through fully connected layers. 3) Calculating the loss: The degree of error is calculate by loss function, by different between actual class label and predicted probability. The aim is to maximize the accuracy and minimize the loss. In our model loss data from epoch 1 is 1.4714, epoch 2 is 0.8175, epoch 3 is 0.612, epoch 4 loss is 0.504 feather more at last epoch 25 loss is 0.1358 . 4) Epoch: The model training includes backward and forward pass across multiple epoch. One complete cycle of an epoch through the entire training dataset. In our model, Training is performed over 25 an epochs with batch\_size of 32. 5) Performance Classification Metrics: [24]Classification metrics determine the performance of deep-learning model for classification tasks. Some commonly used classification metrics. Accuracy is a basic overall evaluation of a model performance. Accuracy is the ratio of accurate prediction instances to the total instances of a dataset. The formula for calculation of accuracy is. Accuracy = (TP+TN)/(TP+FP+TN+FN) Precision and recall are useful element for the evaluation matrix in machine learning to recognize the trade-off between false positive and false negative. Formula for Precision and Recall. Precision=TP/(TP+FP) Recall=TP/(TP+FN) F1-score has the symmetrical means of precision and recall to provide the balance measures. F1-score is an advantage when we are dealing with unevenness datasets, where one class is more persistent than the other[25]. The formula of the F1-score is. F1 Score=2/(1/Precision)+(1/Recall)=2*(Precision*Recall)/ (Precision+Recall). In the whole experiment, we train a tomato leaves disease detection model to improve the tomato crop field production and improve the farmers profit. Total 10,000 sample of tomato leaf images are used to train our model, while training the model accuracy rates goes from epoch 1 be 40\% to around at last epoch 25 got accuracy around 95\% and test accuracy around 88.9\%. While training our model we also lose some data around 1 epoch we lose 1.47 to the last 25 epoch loss be 0.1258 . In the fig4 model accuracy graph between accuracy and epoch, x-axis be an epoch and y-axis be an accuracy. In the fig5 model loss graph between loss and epoch, the x-axis be a epoch and the y-axis be a loss. In fig6 predict outcome result, after training model we compare the images of tomato leaves and give the predicted outcome feather whether the leaves are affected by disease or not and also which type of disease affects the tomato leaves. Example in fig 5 showing predicted:8 be a true:8 means our model identified the correct disease by processing the image

\section{Conclusion}
In our paper we understand the challenges faced by agricultures, to solve their problem we provide some solution to overcome their challenges. In order to increase the crop production we must find the disease faced by plants. We train a model to detect the plant leaf disease. We took a 10,000 tomato plant leaf picture dataset from kaggle public source. After it is divided into three parts 80\% for training, 20\% for validation, 20\% for testing. We use batch\_size 32, epoch size 25, image input size 224x224, approach shear range 0.2, zoom\_range 0.2, horizontal flip true, maxPooling (2,2), dropout layer 0.5 and conv2D(32,(3,3) , activation=’relu’) to (128,(3,3),activation=’relu’). Here we use a small amount of dataset so may be our model less generalizable to the real world problem, still our model accuracy rate be 95.68\% , testing accuracy 88.9\% and validation accuracy 88.81\% . The total size of the train model is 254MB. To improve the model's productiveness and generalization, future study should expand the dataset to improve the range of different types of species of plant and their disease. Overall our research represent the capability of CNN model and Deep learning model to correctly identify and predict the plant illness

\begin{thebibliography}{00}
\bibitem{b1} K. A. Smith, “Why the Tomato Was Feared in Europe for More Than 200 Years,” Smithsonian Magazine, june. 18, 2013. https://www.smithsonianmag.com/arts-culture/why-the-tomat o-was-feared-in-europe-for-more-than-200-years-863735/
\bibitem{b2} “Tomato plant diseases and how to stop them,” USDA, Jul. 16, 2010. https://www.usda.gov/media/blog/2010/07/16/tomato-plant-d iseases-and-how-stop-them
\bibitem{b3} Agarwal, J., Gupta, S., Sharma, N., \& Manchanda, M. (2023). A CNN Method Based Predictive Model for Tomato Leaf Disease Prediction. Proceedings- International Conference on Technological Advancements in Computational Sciences, ICTACS 2023, 262–266. https://doi.org/10.1109/ICTACS59847.2023.1039048
\bibitem{b4} Agarwal, J., Gupta, S., Sharma, N., \& Manchanda, M. (2023). A CNN Method Based Predictive Model for Tomato Leaf Disease Prediction. Proceedings- International Conference on Technological Advancements in Computational Sciences, ICTACS 2023, 262–266. https://doi.org/10.1109/ICTACS59847.2023.10390480
\end{thebibliography}

\end{document}
